\chapter*{Introducción}
\pagenumbering{arabic}
El reconocimiento automático de matrículas (del inglés - \textit{Automatic number plate recognition o ANPR}) es un método de vigilancia en masa que utiliza reconocimiento óptico de caracteres en imágenes para leer las matrículas de los vehículos, Llano et \textit{al} \cite{llano2010sistema}. Esta herramienta es utilizada en muchas tareas, por ejemplo: son utilizadas por la policía para la identificación de vehículos robados o que han cometido alguna infracción; como método de recaudación electrónica de peaje en las autopistas de pago; y, para la vigilancia del tráfico o controlar al acceso a lugares privados, entre otros. El ANPR se puede configurar para almacenar las imágenes, el texto de la matrícula, inclusive es posible almacenar una fotografía del conductor, Llano et \textit{al} \cite{llano2010sistema}. Estos sistemas hacen uso de una alta tecnología en países desarrollados, que hasta utilizan iluminación infrarroja y flashes para hacer posible que la cámara pueda tomar fotografías en cualquier momento del día y a larga distancia. Sin embargo, para países menos desarrollados, la detección y reconocimiento de placas es un reto, ya que no todas las imágenes capturadas con una cámara serán de buena calidad por las condiciones de iluminación, deterioro de las placas y condiciones ambientales. Además, la tecnología ANPR no puede ser copiada de una región a otra, debido a la variación entre formatos de matrículas de cada país, Shivakumara et \textit{al} \cite{Shivakumara2018}.\\

Existe una cantidad significativa de sistemas de reconocimiento de placas con diferentes grados de precisión y velocidad, Du et \textit{al} \cite{Du2013}. Algunos de esos sistemas utilizan el circuito cerrado de televisión existente o radares, y otros son diseñadas específicamente para dicha tarea, Shivakumara et \textit{al} \cite{Shivakumara2018}. Dentro de los sistemas diseñados, se destacan los que usan la técnica de \textit{Deep learning} o aprendizaje profundo, donde la extracción de patrones en los datos estudiados no se realiza manualmente, sino de forma automática. Los métodos más robustos de aprendizaje profundo involucran el uso de redes neuronales convolucionales (RNC) (del inglés - \textit{Convolutional Neural Networks - CNN}). Presentadas por LeCun et \textit{al} \cite{lecun1989backpropagation} a principios de la década de los noventa, las redes convolucionales son un ejemplo de una arquitectura de red neuronal artificial especializada. Específicamente, RNC es un tipo especial de perceptrón multicapa de avance entrenado en modo supervisado utilizando un algoritmo de aprendizaje de retropropagación de descenso de gradiente que permite la extracción automatizada de características, Marin et \textit{al} \cite{marin2013introduccion}.\\ 

Las RNC han probado ser poderosas herramientas para una amplia gama de tareas de visión computarizada, tales como reconocimiento óptico de caracteres, reconocimiento de objetos genéricos, detección de rostros en tiempo real, reconocimiento de voz, reconocimiento de placas, etc, LeCun et \textit{al} \cite{Lecun2015}. RNC automáticamente aprenden abstracciones de nivel medio y de alto nivel obtenidas a partir de datos brutos. Los resultados recientes indican que las características extraídas son extremadamente efectivas en el reconocimiento y localización de objetos en imágenes naturales, Matich et \textit{al} \cite{matich2001redes}. Los inconvenientes de los sistemas de reconocimiento están centrados en la protección de la privacidad y altas tasas de error. Entretanto, los mayores desafíos de las RNC son su alto costo computacional y la demanda de grandes cantidades de muestras para su entrenamiento. Sin embargo, de la mano de los avances tecnológicos (como: grandes almacenamientos de datos y aumento de la potencia informática, sin una gestión activa directa por parte del usuario), estos sistemas han logrado ser mucho más exactos y fiables.\\

En Colombia, hay disponibles muchos sistemas de reconocimiento de placas vehiculares, la mayoría privados y basados en el reconocimiento óptico de caracteres (\textit{Optic Character Recognition - OCR}), ninguno de ellos utiliza redes neuronales convolucionales para generar modelos matemáticos en el reconocimiento de patrones, en este caso, placas vehiculares. Además, aún es posible encontrar que los registros y controles vehiculares se hacen de forma manual, lo que involucra errores constantes y poca seguridad para el usuario, sobretodo en parqueaderos ilegales. En el presente trabajo, se desarrolla y evalúa un método computacional para el reconocimiento de placas de automóviles colombianos. Esta metodología ya ha sido implementada en Colombia para prever patrones en los sistemas financieros (previsiones de precio y tasas cambiarías), sistemas ecológicos (áreas de conservación), minería, de salud e ingeniería. Así está nueva aplicación de una red neuronal convolucional ayudaría en el sistema de tránsito y transportes, ya que permitiría controlar el acceso a parqueaderos, generar tiquetes de pago automáticos, controlar hurtos, entre otras ventajas, como la disminución del costo computacional y el fácil acceso del usuario a este tipo de sistemas.
%===========================================
